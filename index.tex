\documentclass[runningheads,a4paper]{llncs}

\usepackage{amssymb}
\setcounter{tocdepth}{3}
\usepackage{graphicx}

\usepackage{url}
\urldef{\mailsa}\path|{v.bajpai, j.schoenwaelder}@jacobs-university.de|
\newcommand{\keywords}[1]{\par\addvspace\baselineskip
\noindent\keywordname\enspace\ignorespaces#1}

% added by Vaibhav
%----------------------------------------------------------------------
\usepackage[utf8]{inputenc}
\usepackage[T1]{fontenc}
\usepackage[nolist]{acronym}
\renewcommand{\ttdefault}{cmtt}

% temporaries
\usepackage{scrtime}
\usepackage{prelim2e}
\usepackage{todonotes}
\renewcommand{\PrelimWords}{\relax}
\renewcommand{\PrelimText}{\footnotesize[\,\today\ at \thistime\,]}
%----------------------------------------------------------------------


\begin{acronym}
  %\acro{NFQL}{Network Flow Query Language}
\end{acronym}

\begin{document}

\mainmatter  % start of an individual contribution

% first the title is needed
\title{From Global Measurements to Local Management}

\author{Vaibhav Bajpai \and Jürgen Schönwälder%
\thanks{This work was supported by the European Community’s Seventh Framework
Programme (FP7/2007-2013) grant no. 317647 (Leone)}}
\institute{Computer Science, Jacobs University Bremen, Germany\\
\mailsa}
\maketitle

\begin{abstract}
The abstract should summarize the contents of the paper and should
contain at least 70 and at most 150 words. It should be written using the
\emph{abstract} environment.
\keywords{Measurements, Management, IPv6}
\end{abstract}

\section{Introduction} let's see \cite{sundaresan:2011} \ldots

\bibliographystyle{splncs}
\bibliography{index}

\end{document}
