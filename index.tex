\documentclass[a4paper]{llncs}

\usepackage{amssymb}
\setcounter{tocdepth}{3}
\usepackage{graphicx}

\usepackage{url}
\urldef{\mailsa}\path|{v.bajpai, j.schoenwaelder}@jacobs-university.de|
\newcommand{\keywords}[1]{\par\addvspace\baselineskip
\noindent\keywordname\enspace\ignorespaces#1}

% added by Vaibhav
%----------------------------------------------------------------------
\usepackage[utf8]{inputenc}
\usepackage[T1]{fontenc}
\usepackage[nolist]{acronym}
\renewcommand{\ttdefault}{cmtt}
\usepackage{cite}

% temporaries
\usepackage{scrtime}
\usepackage{prelim2e}
\usepackage{todonotes}
\renewcommand{\PrelimWords}{\relax}
\renewcommand{\PrelimText}{\footnotesize[\,\today\ at \thistime\,]}
%----------------------------------------------------------------------


\begin{acronym}
  \acro{RIR}{Regional Internet Registry}
  \acro{NETCONF}{Network Configuration}
  \acro{CGN}{Carrier-Grade NAT}
  \acro{NAT}{Network Address Translation}
  \acro{FCC}{Federal Communications Commission}
  \acro{RIPE NCC}{RIPE Network Coordination Centre}
  \acro{RTT}{round-trip time}
  \acro{M-Lab}{Measurement Lab}
  \acro{CDN}{Content Delivery Network}
\end{acronym}

\begin{document}

\mainmatter  % start of an individual contribution

% first the title is needed
\title{Using Global Measurements to Understand the Evolution of the Internet}

\author{Vaibhav Bajpai \and Jürgen Schönwälder%
\thanks{This work was supported by the European Community’s Seventh Framework
Programme (FP7/2007-2013) grant no. 317647 (Leone)}}
\institute{Computer Science, Jacobs University Bremen, Germany\\
\mailsa}
\maketitle

\begin{abstract}
The abstract should summarize the contents of the paper and should
contain at least 70 and at most 150 words. It should be written using the
\emph{abstract} environment.
\end{abstract}


% sections
%----------------------------------------------------------------------
\section{Research Statement}
The curiosity to understand the evolution of the Internet from the user'
vantage point started by establishing techniques to remotely probe the
broadband access network. Dischinger \emph{et al.} in \cite{dischinger:2007},
for instance, inject packet trains and use the responses received from
residential gateway to infer the broadband link characteristics. This led to
the development of a number of software-based solutions such as
\texttt{netalyzr} \cite{kreibich:2010}, that require explicit interactions
with the broadband consumer.  Recently, the requirement for accurate
measurements, coupled with \ac{FCC}' initiated efforts to define data-driven
standards, has led to the deployment of a number of large-scale measurement
platforms that perform measurements using dedicated hardware probes not only
from within the \ac{ISP}' network but also directly from the home gateway.

In a recent study, sponsered by the \ac{FCC}, Sundaresan \emph{et al.}
\cite{sundaresan:2011} have used such a measurement data from a swarm of
deployed SamKnows probes to investigate the throughput and latency of access
network links across multiple \ac{ISP}s in the United States. They have
coupled this data with their own Bismark platform \cite{sundaresan:2012} to
investigate different traffic shaping policies enforced by the \ac{ISP} and to
understand the bufferbloat phenomenon. The empirical findings of this study
has recently been repraised by Canadi \emph{et al.} in \cite{canadi:2012}
where they use crowdsourced data from \url{speedtest.net} to compare both
results. The primary aim of all these activities is to measure the performance
and reliability of broadband access networks and facilitate the regulators
with research findings to help them make policy decisions
\cite{draft-schulzrinne-lmap-requirements-00}.  As a result, the focus is on
defining metrics and implement measurement tests that help achieve this goal.

Using a large-scale measurement platform we want to take this further and
study the evolution of the Internet. We want to define metrics and implement
measurement tests that help us answer questions of the form:

\begin{itemize}
  \item \emph{How does the performance of IPv6 compare to that of IPv4 in the real world?}
  \item \emph{Can we identify a \ac{CGN} from a home gateway?}
  \item \emph{Can we identify multiple layers of NAT from a home gateway?}
  \item \emph{How much do web services centralize on \ac{CDN}s?}
  \item \emph{To what extend does the network experience depend on regionalization?}
\end{itemize}

In the past, we have performed an experimental evaluation of IPv6
transitioning technologies to identify how well current applications and
protocols interoperate with them \cite{vbajpai:2012}. We are now participating
in the Leone\footnote{\url{http://leone-project.eu}} project whose primary
goal is to define metrics and implement tests that can asses the end-user
quality of experience by analyzing data collected from several measurements
running on SamKnows probes.
\label{sec:rstatement}
\section{Proposed Approach}
% samknows and bismark
SamKnows \footnote{\url{http://www.samknows.com}} specializes in the
deployment of hardware-based probes that perform measurements to assess the
performance of broadband access networks. The probes function by performing
active measurements when the user is less aggresively using the network.  RIPE
Atlas \footnote{\url{https://atlas.ripe.net}} is another independent
measurement infrastructure deployed by \ac{RIPE NCC}. It consists of thousands
of probes distributed around the globe that perform \ac{RTT} and traceroute
measurements to a number of preconfigured destinations alongside DNS queries
to root DNS servers.

% measurement lab
\ac{M-Lab} \cite{dovrolis:2010} is an open, distributed platform to deploy
internet measurement tools and the resulting measurement data on Google'
storage infrastructure. The tools vary from measuring TCP throughput and
available bandwidth to emulating clients to identify end-user traffic
differentiation policies \cite{dischinger:2010, kanuparthy:2011} to performing
reverse traceroute lookups from arbitrary destinations \cite{bassett:2010}.
All of the collected data is available in the public domain.

% leone

The answers to the aforementioned research questions will only materialize
with access to an available large-scale measurement platform. We as partners
of the Leone \footnote{\url{http://leone-project.eu}} consortium, will
leverage the infrastructure of our partners: BT
\footnote{\url{http://www.bt.com}}, TI
\footnote{\url{http://www.telecomitalia.com}} and SamKnows. The developed
measurement tests will be deployed not only in our partner's ISP's network,
but also in the already deployed SamKnows global infrastructure.  The SamKnows
infrastructure already has several thousand probes in the homes of broadband
customers, and will continue to grow during the STREP lifetime.
\label{sec:approach}
\section{Preliminary Results}
\begin{figure}[t]
  \begin{minipage}[t]{0.50\textwidth}
    \centering
    \resizebox*{1.0\textwidth}{!}{\includegraphics{figures/t28971-competition-300ms}}
    \caption{Native IPv4 and Teredo Tunnel}
  \end{minipage}
  \begin{minipage}[t]{0.50\textwidth}
    \centering
    \resizebox*{1.0\textwidth}{!}{\includegraphics{figures/unimator2-competition-300ms}}
    \caption{Native IPv4 and Native IPv6}
  \end{minipage}
\caption{\label{fig:happy-v4-v6-compete}IPv4 and IPv6 Happy Eyeball Competition} 
\end{figure}

A dual-stacked user when attempting to connect to a dual-stacked service
traditionally prefers connecting over IPv6. This is because in POSIX systems,
the domain name resolution system call \texttt{getaddrinfo(\ldots)} returns a
list of endpoints in an order that prioritizes an IPv6-upgrade path
\cite{rfc6724}. The dictated order can dramatically reduce the application
responsiveness in situations where IPv6 connectivity is broken. This is
because, the attempt to connect over an IPv4 endpoint will take place only
when the IPv6 connection attempt has timed out, which can be in the order of
seconds.

\begin{figure}[t]
  \begin{minipage}[t]{0.50\textwidth}
    \centering
    \resizebox*{1.0\textwidth}{!}{\includegraphics{figures/t28971-mean}}
    \caption{Native IPv4 and Teredo Tunnel}
  \end{minipage}
  \begin{minipage}[t]{0.50\textwidth}
    \centering
    \resizebox*{1.0\textwidth}{!}{\includegraphics{figures/unimator2-mean}}
    \caption{Native IPv4 and Native IPv6}
  \end{minipage}
\caption{\label{fig:happy-v4-v6-mean-std} Mean and Standard Deviations for
IPv4 and IPv6}
\end{figure}

This noticeable degraded user experience can be subverted by making
applications apply the happy eyeballs algorithm \cite{rfc6555}. The algorithm
recommends that a dual-stacked application resolves the DNS names of a
dual-stacked service for both IPv4 and IPv6 endpoints at once. If the resolver
returns both endpoints, the application must try a TCP
\texttt{connect(\ldots)} to both the endpoints pick the one that
completes first. The connection attempt does prefer the first resolved
address family (usually IPv6) by the order of 300ms though.

In this pursuit, to determine whether applications will use IPv4 or IPv6 on a
dual stacked service, we developed \texttt{happy}, a simple TCP happy eyeballs
probing tool. It uses non-blocking \texttt{connect(\ldots)} calls to establish
concurrent connections to a number of possible endpoints of a service.
%The tool, however, does not check whether the endpoints of a given target all
%provide the same service. Hence, it is possible to impact the results by
%setting up fake servers that do not provide the service tested and which are
%designed and deployed with the only purpose to provide fast connection setup
%times and redirect services.
We have cross-compiled \texttt{happy} for the OpenWRT
\footnote{\url{https://openwrt.org}} platform. As a result, the tool can now
be run on widely deployed SamKnows probes and the collected measurement data
can be further analysed. In order to ascertain the value in this exercise, we
prepared an internal test-bed of multiple measurement points. The measurement
points have different flavors of IPv4 and IPv6 connectivity ranging from
native IPv4, native IPv6, IPv6 tunnel broker endpoints, Teredo and tunnelled
IPv4. We used the top 100 DNS names compiled by Hurricane Electric Internet
Services \footnote{\url{http://bgp.he.net/ipv6-progress-report.cgi}} and ran
\texttt{happy} on the set of dual-stack services represented by these DNS
names.

\begin{figure}[t]
  \begin{minipage}[t]{0.50\textwidth}
    \centering
    \resizebox*{1.0\textwidth}{!}{\includegraphics{figures/happy-v4cloud}}
  \end{minipage}
  \begin{minipage}[t]{0.50\textwidth}
    \centering
    \resizebox*{1.0\textwidth}{!}{\includegraphics{figures/happy-v6cloud}}
  \end{minipage}
\caption{\label{fig:v4-v6-cloud}IPv4 and IPv6 aggregation cloud}
\end{figure}

A preliminary result comparing the preference of a happy-eyeballed application
to IPv6 and IPv4 from two measurement points is shown in Fig.
\ref{fig:happy-v4-v6-compete}. The initial results show that happy eyeballs
prevents IPv6 access to Facebook, with only a 20\% chance to get to Google
related services over a Teredo Tunnel. The results look more promising on a
native IPv6 connection. It is important to note that adhering to the happy
eyeballs recommendation, IPv6 endpoints are allowed a 300ms chance to succeed.
A result comparing the time (mean and standard deviation) to make a TCP
connection to each of the services from the same measurement points shown in
Fig. \ref{fig:happy-v4-v6-mean-std}. The initial results show higher time
variances when connections are made over IPv6. In addition, it appears, some
of the related (and few of the unrelated) services show similar preferences.
These services either resolve to the same endpoint or a set of endpoints that
belong to the same allocated endpoint. Digging through the \texttt{whois}
information for each of the endpoints from their \ac{RIR} seems to indicate
that major portion of the services map to allocated prefixes owned by popular
organizations like Google and Akamai Technologies as shown in Fig.
\ref{fig:v4-v6-cloud}.

%As much as it is important to define and implement new tests on these
%measurement infrastructure, it is also equally pertinent to not only be able
%to install, update and delete these tests but also configure the entire suite
%of probes using a standardized protocol over the network. The \ac{NETCONF}
%protocol \cite{rfc6241} is particularly designed to cater to this problem.
%Towards this end, we have built a \ac{NETCONF} server for the OpenWRT platform
%using the \texttt{libnetconf}
%\footnote{\url{http://code.google.com/p/libnetconf/}} library and tested the
%implementation using our NETCONF Python API \texttt{ncclient}
%\cite{sbhushan:2009}. This will allow automated deployment of measurement
%tests and remote management of their startup configurations.

\label{sec:preliminaryresults}
%----------------------------------------------------------------------



% bibliography
%----------------------------------------------------------------------
\bibliographystyle{splncs}
\bibliography{index}
%----------------------------------------------------------------------

\end{document}
