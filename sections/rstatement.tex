The curiosity to understand the performance of the Internet from the user's
vantage point led to the development of techniques to remotely probe broadband
access networks. Dischinger \emph{et al.} in \cite{dischinger:2007}, for
instance, inject packet trains and use the responses received from residential
gateways to infer broadband link characteristics. This led to the development
of a number of software-based solutions such as \texttt{netalyzr}
\cite{kreibich:2010}, that require explicit interactions with the broadband
customer. Recently, the requirement for accurate measurements, coupled with
efforts initiated by regulators to define data-driven standards, has led to
the deployment of a number of large-scale measurement platforms that perform
measurements using dedicated hardware probes not only from within \ac{ISP}
networks but also directly from home gateways.

In a recent study, sponsered by the \ac{FCC}, Sundaresan \emph{et al.}
\cite{sundaresan:2011} have used measurement data from a swarm of deployed
SamKnows probes to investigate the throughput and latency of access network
links across multiple \ac{ISP}s in the United States. They have analyzed this
data together with data from their own Bismark platform \cite{sundaresan:2012}
to investigate different traffic shaping policies enforced by \ac{ISP}s and to
understand the bufferbloat phenomenon. The empirical findings of this study
have recently been repraised by Canadi \emph{et al.} in \cite{canadi:2012}
where they use crowdsourced data from \url{speedtest.net} to compare both
results. The primary aim of all these activities is to measure the performance
(e.g., bandwidth, latency, jitter) and reliability of broadband access
networks and facilitate the regulators with research findings to help them
make policy decisions \cite{draft-schulzrinne-lmap-requirements-00}.

Using a large-scale measurement platform we want to take this further and
study the impact of network infrastructure changes. We want to define metrics,
implement measurement tests and data analysis tools that help us answer
questions of the form:

\begin{itemize}
  \item \emph{How does the performance of IPv6 compare to that of IPv4 in the real world?}
  \item \emph{Can we identify a \ac{CGN} from a home gateway?}
  \item \emph{Can we identify multiple layers of NAT from a home gateway?}
  \item \emph{How much do web services centralize on \ac{CDN}s?}
  \item \emph{To what extend does the network experience depend on regionalization?}
\end{itemize}

In the past, we have performed an experimental evaluation of IPv6
transitioning technologies to identify how well current applications and
protocols interoperate with them \cite{vbajpai:2012}. We are now participating
in the Leone\footnote{\url{http://leone-project.eu}} project, whose primary
goal is to define metrics and implement tests that can asses the end-user's
quality of experience by analyzing data collected from measurements running on
SamKnows probes.
