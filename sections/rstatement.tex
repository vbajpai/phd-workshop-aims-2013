An interest to understand the evolution of the Internet from the user' vantage
point started with establishing techniques to remotely probe the broadband
links. Dischinger \emph{et al.} in \cite{dischinger:2007} for instance, inject
packet trains and use the responses received from gateway to infer the
broadband link characteristics. This led to development of a number of
software-based solutions, \texttt{netalyzr} \cite{kreibich:2010}, for
instance, that require explicit interactions with the broadband consumer.
Recently, the requirement for accurate measurements, coupled with \ac{FCC}'
initiated efforts to define data-driven standards has led to the deployment of
a number of large-scale measurement platforms that perform measurements using
dedicated hardware probes not only from within the ISP' network but also
directly from the home gateway. 

In a recent study, sponsered by the \ac{FCC}, Sundaresan \emph{et al.}
\cite{sundaresan:2011} have used such a measurement data from a swarm of
deployed SamKnows probes to investigate the throughput and latency of access
network links across multiple ISPs. in the United States. They have coupled
this data with their own Bismark platform \cite{sundaresan:2012} to also
investigate different traffic shaping policies enforced by the ISP and to
understand the bufferbloat phenomenon \cite{gettys:2012}.  The empirical
findings of this study has recently been repraised by Canadi \emph{et al.} in
\cite{canadi:2012} where they use crowdsourced data from \url{speedtest.net}
to compare both results. The primary aim is to measure the performance and
reliability of broadband access networks and facilitate the regulators with
research findings to help them make policy decisions that eliminate the ISP'
monopoly \cite{draft-schulzrinne-lmap-requirements-00}.  As a result, the
focus is on defining metrics and implement measurement tests that help achieve
this goal.

In the past, we have performed an experimental evaluation of IPv6
transitioning technologies to identify how well current applications and
protocols interoperate with them \cite{vbajpai:2012}. Using a large-scale
measurement platform we want to take this further and study the IPv6 evolution
and its repercussions. We want to describe metrics and implement measurement
tests that help us answer questions of the form:

\begin{itemize}
  \item \emph{How does the performance of IPv6 compare to that of IPv4 in real world?}
  \item \emph{Can we identify deployment of \ac{CGN} from a gateway?}
  \item \emph{Can we identify deployments of multiple layers of NAT from a gateway?}
  \item \emph{How much do services centralize on core \ac{CDN}s?}
  \item \emph{To what extend does the network experience depend on
              localized data?}
\end{itemize}

%In our preliminary study, we have witnessed that a major portion of the
%services in practicality centralize either on core content delivery networks
%or major cloud platforms.
