An interest to understand the evolution of the Internet from the user' vantage
point started with establishing techniques to remotely probe the broadband
links. Dischinger \emph{et al.} in \cite{dischinger:2007} for instance, inject
packet trains and use the responses received from gateway to infer the
broadband link characteristics. This led to development of a number of
software-based solutions, \texttt{netalyzr} \cite{kreibich:2010}, for
instance, that requires explicit interactions with the broadband consumer.
Recently, the requirement for accurate measurements, coupled with \ac{FCC}'
initiated efforts to define data-driven standards has led to the deployment of
a number of large-scale measurement platforms that perform measurements using
dedicated hardware probes not only from within the ISP' network but also
directly from the home gateway.

% samknows and bismark
SamKnows \footnote{\url{http://www.samknows.com}} specializes in such
measurements to study the performance of broadband access networks. It
functions by deploying dedicated hardware probes in the home gateway, that
perform active measurements when the user is less aggresively using the
network. In a recent study, sponsered by the \ac{FCC}, Sundaresan \emph{et
al.} \cite{sundaresan:2011} have used this measurement data to investigate the
throughput and latency of access network links across multiple ISP's. in the
United States. They have coupled this data with their own Bismark platform
\cite{sundaresan:2012} to investigate different traffic shaping policies
enforced by the ISP and understand the bufferbloat phenomenon
\cite{gettys:2012}. The empirical findings of this study has recently been
repraised by Canadi \emph{et al.} in \cite{canadi:2012} where they use
crowdsourced data from \url{speedtest.net} to compare both results.

% ripe atlas
Ripe Atlas \footnote{\url{https://atlas.ripe.net}} is another independent
measurement infrastructure deployed by \ac{RIPE NCC}. It consists of thousands
of probes distributed around the globe that perform \ac{RTT} and traceroute
measurements to a number of preconfigured destinations alongside DNS queries
to root DNS servers.

% measurement lab
\ac{M-Lab} \cite{dovrolis:2010} is an open, distributed platform to deploy
internet measurement tools and the resulting measurement data on Google'
storage infrastructure. The tools vary from measuring TCP throughput and
available bandwidth to emulating clients to identify end-user traffic
differentiation policies \cite{dischinger:2010, kanuparthy:2011} to performing
reverse traceroute lookups from arbitrary destinations \cite{bassett:2010}.
All of the collected data is available in the public domain.
