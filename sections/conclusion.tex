Using our \texttt{happy} eyeballs probing tool we have performed a preliminary
study on IPv6. Using a large-scale measurement platform we want to take this
further and define new metrics and measurement tests that help us uncover
more insights into the evolution and inner workings of the Internet.


%In our preliminary study, we have witnessed that a major portion of the
%services in practicality centralize either on core content delivery networks
%or major cloud platforms. We want to investigate this effect in more detail
%and understand to what extend does this network aggregation and the eventual
%user experience depend on the localization information. We want to take this
%further by comparing the performance of IPv6 with respect to IPv4 and not only
%define IPv6 related metrics but also identify \ac{CGN}s and/or several layers
%of \ac{NAT}s enforced by the \ac{ISP} on the home gateway.

%%Our current NETCONF server implementation for the SamKnows platform assumes
%%the existence of an ssh server implementation that provides subsystem support.
%%We would like to subvert this limitation and instead secure NETCONF exchanges
%%over TLS \cite{draft-ietf-netconf-rfc5539bis-01}. We are also interested to
%%define new YANG \cite{rfc6020} data models for configuring and scheduling the
%%measurement tests for such large-scale broadband access measurements. These
%%data models can be used with either \ac{NETCONF} or using a RESTful interface
%%\cite{draft-bierman-netconf-yang-api-01}.
