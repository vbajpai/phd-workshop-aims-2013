% samknows and bismark
SamKnows\footnote{\url{http://www.samknows.com}} specializes in the deployment
of hardware-based probes that perform measurements to assess the performance
of broadband access networks. The probes function by performing active
measurements when the user is not aggressively using the network.  RIPE
Atlas\footnote{\url{https://atlas.ripe.net}} is another independent
measurement infrastructure deployed by the \ac{RIPE NCC}. It consists of
thousands of hardware probes distributed around the globe that perform
\ac{RTT} and traceroute measurements to a number of preconfigured destinations
alongside DNS queries to DNS root servers.

% measurement lab
\ac{M-Lab} \cite{dovrolis:2010} is an open, distributed platform to deploy
internet measurement tools. The measurement results are stored on Google'
infrastructure. The tools vary from measuring TCP throughput and available
bandwidth to emulating clients to identify end-user traffic differentiation
policies \cite{dischinger:2010, kanuparthy:2011} to performing reverse
traceroute lookups from arbitrary destinations \cite{bassett:2010}.

% leone
The answers to the aforementioned research questions will only materialize
with access to a large-scale measurement platform. We as partners of the Leone
consortium, will leverage the infrastructure of our partners. We will define
metrics particularly targetted to our research questions and complement them
by implementing subsequent measurement tests. The developed measurement tests
will be deployed not only in our partner \ac{ISP}'s network, but also in the
already deployed SamKnows global infrastructure.  The SamKnows infrastructure
already has several thousand deployed probes and will continue to grow during
the project' lifetime. The data will be collected for a time long enough to be
representative of a globally spanned network, devoid of localized anomalies.
The collected data will be conglomerated from multiple measurement points and
finally analyzed to uncover information needed to help us answer these
questions. We have started with a study to assess the growth of IPv6 in the
real world and have uncovered few insights that we discuss in the next
section.
