In the past, we performed an experimental evaluation of IPv6 transitioning
technologies to identify how well current applications and protocols
interoperate in such a deployement scenario \cite{vbajpai:2012}. In the
future, we want to study and compare the performance of IPv6 with respect to
IPv4 from a dual-stacked home gateway.  We are also interested to define
metrics that can identify whether the home gateway is behind a \ac{CGN}  or is
otherwise encompassed by several layers of \ac{NAT}s enforced by the ISP.

In our preliminary study, we have witnessed that a major portion of the
services in practicality centralize either on core content delivery networks
or major cloud platforms. We want to investigate this effect in more detail
and understand to what extend does this network aggregation and the eventual
user experience depend on the localization information.

Our current NETCONF server implementation for the SamKnows platform assumes
the existence of an ssh server implementation that provides subsystem support.
We would like to subvert this limitation and instead secure NETCONF exchanges
over TLS \cite{draft-ietf-netconf-rfc5539bis-01}. We are also interested to
define new YANG \cite{rfc6020} data models for configuring and scheduling the
measurement tests for such large-scale broadband access measurements. These
data models can be used with either \ac{NETCONF} or using a RESTful interface
\cite{draft-bierman-netconf-yang-api-01}.
