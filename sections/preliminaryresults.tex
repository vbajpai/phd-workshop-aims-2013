\subsection{Measurements}

A dual-stacked user when attempting to connect to a dual-stacked service
traditionally prefers connecting over IPv6. This is because in POSIX systems,
the internal domain name resolution system call \texttt{getaddrinfo(\ldots)}
\cite{rfc3493} returns the list of addresses in an order that prioritizes an
IPv6-upgrade path \cite{rfc3484}. The dictated order can dramatically reduce
the application responsiveness in situations where IPv6 connectivity is
broken. This is because, the attempt to connect over an IPv4 address will take
place only when the IPv6 connection attempt has timed out, which can be in the
order of seconds.

This noticeable degraded user experience can be subverted by making
applications apply the happy eyeballs algorithm \cite{rfc6555}. The algorithm
recommends that a dual-stacked application try resolving a dual-stacked
service for both IPv4 and IPv6 addresses at once. If the resolver returns both
addresses, the application must try a TCP \texttt{connect(\ldots)} to both the
resolved addresses and pick the one that completes first.

In this pursuit, to determine whether applications will use IPv4 or IPv6 on a
dual stacked service, we developed \texttt{happy}, a simple TCP happy eyeballs
probing tool. It uses non-blocking \texttt{connect(\ldots)} calls to establish
concurrent connections to a number of possible endpoints of a service. The
tool does not check whether the endpoints of a given target all provide the
same service. Hence, it is possible to impact the results by setting up fake
servers that do not provide the service tested and which are designed and
deployed with the only purpose to provide fast connection setup times.

We have cross-compiled \texttt{happy} for the OpenWRT \cite{fainelli:2008}
platform. As a result, the tool can now be run on widely deployed SamKnows
probes \footnote{\url{http://www.samknows.com}}, and the collected measurement
data can be further analysed. In order to ascertain the value in this
exercise, we prepared an internal test-bed of multiple measurement points. The
measurement points have different flavors of IPv4 and IPv6 connectivity
ranging from native IPv4, native IPv6, 6in4, Teredo \cite{rfc4380} and
tunnelled IPv4.

\begin{figure}[t]
\centering
\includegraphics*[width=1.0\linewidth]{figures/happy-servers-mean-std}
\caption{servers vs \texttt{\{mean, std\}}}
\label{fig:v6cloud}
\end{figure}

\begin{figure}
  \begin{minipage}[h]{0.50\textwidth}
    \centering
    \resizebox*{1.0\textwidth}{!}{\includegraphics{figures/happy-v4cloud}}
  \end{minipage}
  \begin{minipage}[h]{0.50\textwidth}
    \centering
    \resizebox*{1.0\textwidth}{!}{\includegraphics{figures/happy-v6cloud}}
  \end{minipage}
\caption{\label{fig:v4-v6-cloud}IPv4 and IPv6 aggregation cloud}
\end{figure}

\subsection{Local Management}
